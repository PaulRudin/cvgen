\documentclass[11pt,a4paper]{article}
\usepackage[inline]{enumitem}
\setlist[itemize]{topsep=6pt}
\usepackage{titlesec}
\titleformat{\subsection}[block]{\Large\bfseries\filcenter}{}{1em}{}
\usepackage[nohead,nofoot,hmargin=2cm,vmargin=1cm]{geometry}
\usepackage{tabularx}
\usepackage{wasysym}
\usepackage{marvosym}
\usepackage{hyperref}
\usepackage{mathptmx}
\usepackage{xcolor}
\hypersetup{
    colorlinks,
    linkcolor={red!50!black},
    citecolor={blue!50!black},
    urlcolor={blue!80!black}
}

\setlength\parindent{0pt}
%course/job title ; employer/institution ; dates
\newcommand{\centry}[3]{\paragraph{#1} \textit{#2}%
\hfill#3\\[2pt]}
%
\renewcommand{\familydefault}{\sfdefault}


\begin{document}
\begin{tabularx}{\linewidth}{XX}
  \begin{flushleft}
    {\large PAUL RUDIN}\\[\baselineskip]
    \href{https://github.com/PaulRudin}{https://github.com/PaulRudin}\\
    \href{https://www.linkedin.com/in/paulrudin/}{https://www.linkedin.com/in/paulrudin}
  \end{flushleft}
&
  \begin{flushright}
    \href{tel:+447939720169}{+44 7939 720169}\\
    \href{mailto:paul@rudin.co.uk}{paul@rudin.co.uk}\\[\baselineskip]
    2 Sedley Taylor Road,\\
    Cambridge, CB2 8PW
  \end{flushright}
\end{tabularx}

\rule{\textwidth}{1pt}

\subsection*{Employment}

\centry{Full Stack App Developer}{Xamaral Limited / Self Employed}{Mar 2021--present}

I work on Android, IOS and web apps for displaying real time data for the
Fantasy Premier League game. The back end stack is deployed to GCP using GKE,
the Firebase Real time database, Cloud Storage, Cloud Pubsub, Cloud Scheduler,
Cloud Functions and Cloud Run. The stack includes for data acquisition and
ingestion, and high performance scoring and ranking functionality utilising
Python/Cython and Numpy, which is made available for front end applications via
Graphql subscriptions. Components are mostly implemented in Python, but some
use Node.js. Front end applications are implemented in Flutter (Dart) and are
available in the Google Play Store and Apple App Store.


\centry{Senior Devops Engineer}{Ryff Limited}{Dec 2020--Mar 2021}

I developed Terraform configuration allowing for the deployment of separate
production, development, staging and test environments for the company's
software stack to AWS.

\centry{Principal Devops Engineer}{University of Cambridge}{May
  2020--Oct 2020}

I designed and implemented cloud deployment strategies. I
used Gitlab pipelines and Terraform to create and update GCP
resources. I also developed new and existing web applications and APIs
in Python (Django and Django Rest Framework), as well as tooling for
packaging as Docker images.

\centry{Software Development Engineer}{Speechmatics Ltd.}{Oct 2018--June 2019}

I worked as part of the team delivering a new SaaS speech transcription
service. The system consists of a set of interacting microservices implemented
in Python and Golang. We used Elasticsearch, Logstash and Kibana for analytics
and support tools. We used Terraform to provision cloud infrastructure on Azure
and Jsonnet to generate consistent Terraform and Kubernetes
configuration. We used ArgoCD to keep deployed resources in sync with
git repos, and Prometheus, Elasticsearch, Logstash and Kibana for
monitoring and dashboards.

\centry{Principal Software Engineer}{Cambridge Consultants Ltd.}{July 2018--Oct
2018}

I worked as part of the Internet Software and Services Group, delivering
bespoke cloud-based solutions for clients, deployed to Google Cloud. My focus
was on creating back end api servers and orchestration of interacting
components in kubernetes clusters.

\centry{Senior Software Engineer}{Grapeshot Ltd. (now part of
  Oracle)}{2016--July 2018}

I was responsible for a developer portal and a suite of associated
REST apis. The portal is a Python/Django application and provides
account management, subscription plans, quota and usage enforcement
and associated documentation. The apis are implemented using either
Node.js/Hapi.js or Python/Aiohttp. Components are packaged as Docker
images and deployed to Kubernetes clusters (configured using
Ksonnet/Jsonnet) provisioned on AWS. Persistence was achieved using
MySql and Elasticsearch. Jenkins is used to automate building, testing
and deployment. We provide SDKs and code samples in various languages
to facilitate and illustrate use of the apis. We use scrum and agile
software development practices.

\centry{Contracts Manager}{University of Cambridge}{2015--2016}
I negotiated and drafted research contracts on behalf of the university.

\centry{Paralegal}{Ministry of Justice/Sharpe Pritchard LLP}{2014--2015} I
assisted the Ministry in negotiations on a large IT contract.

\centry{Freelance Computer Programmer}{Flax}{2005--2013} I designed and
implemented web-based search systems for a variety of clients, usually as a
subcontractor for \href{http://www.flax.co.uk}{Flax}. I completed stand-alone
projects in their entirety as well as working as part of a team on larger
projects.  Most projects were implemented in Python, using the Xapian or Lucene
search engines. I made significant contributions to a high level Python wrapper
for Xapian; wrote an off-the-shelf open source search system from scratch; as
well as many modules for text processing and tokenisation; web, database and
file system crawling; indexing and search, as well as web front ends.

\centry{Computer Programmer and Team Leader}{Scientia Ltd.}{1999--2005} I was
responsible for the development of the company's flagship academic scheduling
software. This work involved: designing and implementing timetable scheduling
algorithms and heuristics; supervising other programmers; liaising with other
parts of the company and clients on new features; and determining appropriate
bug-tracking, testing and source control processes. I recruited and led other
programmers.


\centry{Research Associate}{Queen Mary College}{1994}
I worked in the computer science department on research and implementation in
the field proof theory. I investigated the automatic generation of natural
deduction proofs.

\centry{Assistant Director, Financial Analyst, Computer Programmer}{MDT
  Limited}{1986-1992}
I developed computer models for assessing the value of large, complex
funding transactions (e.g. bond issues, aircraft leases). I was involved in
negotiations with clients, funders and legal advisers; preparing contracts;
writing proposals and numerical analyses of transactions.


\subsection*{Education}

\centry{Online Courses}{\emph{via} Udacity or Coursera}{2011--2015} I have undertaken several online courses
on theoretical and practical aspects of computing. In all cases I attained high
grades.
\begin{enumerate*}[label=(\alph*)]
\item John Hopkins data science courses:
  \href{https://drive.google.com/open?id=0BzIgco934WByWEt1RVdxazVSckk}{Data
    Scientist's Toolbox},
  \href{https://drive.google.com/open?id=0BzIgco934WByeWdrM0pNMndVS2M}{Getting
    and Cleaning Data},
  \href{https://drive.google.com/open?id=0BzIgco934WBydkxiNGJTUFNCQVU}{R
    Programming},
  \href{https://drive.google.com/open?id=0BzIgco934WByTmtVYjl3MXROZG8}{Exploratory
    Data Analysis}.
\item Stanford courses:
  \href{https://drive.google.com/open?id=0BzIgco934WByR3NldDVjVTFfR2s}{Natural
    Language Processing},
  \href{https://drive.google.com/open?id=0BzIgco934WByZ0VLRm1KWjlTaGM}{Algorithms:
    Design and Analysis},
  \href{https://drive.google.com/open?id=0BzIgco934WBydEY5TWJVcmltREk}{Introduction
    to Artificial Intelligence}, \href{https://drive.google.com/open?id=0BzIgco934WByZ0VLRm1KWjlTaGM}{Machine Learning}.
\item UC Berkley courses: \href{https://drive.google.com/open?id=0BzIgco934WByZ0VLRm1KWjlTaGM}{Software as a Service}, \href{https://drive.google.com/open?id=0BzIgco934WByRDJJYWpCZkx0OHc}{Web Development}.
\end{enumerate*}

\centry{Bar Professional Training Course}{BBP Law School}{2013--2014}
 Passed graded Very Competent. I have been called to the bar. Transcript:
\url{http://goo.gl/wSo4yn}


\centry{Graduate Diploma in Law}{BPP Law School}{2013} 
Passed with Distinction. Transcript: \url{http://goo.gl/ljZMYs}.

\centry{Doctorate in Theoretical Computer Science}{St Hugh's College, Oxford
  University} {1994--1998} My thesis is entitled ``A framework for diagrammatic
reasoning''. I use category theory to develop techniques for using diagrams and
their transformations as formal mathematical proofs.  I taught programming to
engineering undergraduates (mostly C++).  Awarded an Engineering and Physical
Sciences Research Council Studentship.


\centry{MSc, Advanced Methods in Computer Science}{Queen Mary College, London
  University}{1992--1993} Passed with distinction, with distinction grades in
every paper (transcript: \url{https://goo.gl/i1wdJ5}) The course included the
modules: Automated Reasoning; Expert Systems; Logic for Knowledge
Representation and Reasoning; Human Factors for Interactive System Design.  My
dissertation investigated using genetic algorithms to learn logic programs and
included software implement in Prolog.


\centry{BA and MA, Computer Science}{Pembroke College, Cambridge University}{1983--1986}
I undertook a variety of theoretical and practical computing courses.

\subsection*{Technical Skills}

I have used a wide variety of technologies. I research new techniques and
technologies in order to make architectural decisions. Recent projects use the
following languages, technologies and development tools: Python (Django,
Aiohttp), Numpy, Pandas, Cython, Node.js (Hapi.js), GraphQL, Elasticsearch,
Kibana, Logstash, Filebeats, Redis, RabbitMq, MySQL, Nginx, shell scripts,
Make, linux, HTML, CSS, Json, \LaTeX, Docker.  Git, Jenkins, Bitbucket, Github,
Gitlab, Jira, Confluence. The main cloud platforms - AWS, GCP and
Azure. Kubernetes and associated tools. Flutter, Dart and Firebase. I have a
good understanding of linux, internet protocols and networking. I am familiar
with agile software development and have a good understanding of Scrum.


\end{document}
